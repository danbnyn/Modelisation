\documentclass{article}
\usepackage[utf8]{inputenc}
\usepackage{amsmath}
\usepackage{hyperref}

\title{État de l'art: Virothérapie dans le traitement du cancer}
\author{Votre nom}
\date{Date}

\begin{document}

\maketitle

\section*{Introduction}

La virothérapie, utilisée comme immunothérapie contre le cancer, a démontré son potentiel mais fait face à des défis majeurs.

\section*{État de l'art}

Les virus oncolytiques, conçus spécifiquement pour cibler les cellules cancéreuses, rencontrent des mécanismes de résistance qui peuvent réduire leur efficacité \cite{goradel2022}. La compréhension de ces mécanismes est cruciale pour améliorer les stratégies thérapeutiques actuelles. Les recherches indiquent que les virus oncolytiques peuvent être utilisés efficacement dans différentes approches thérapeutiques, y compris en combinaison avec la chimiothérapie pour traiter divers types de cancer \cite{zeyaullah2012}.

Les défis liés à la sélection du virus et de la plateforme génétique appropriés sont importants. La sélection du virus idéal pour une thérapie donnée implique de prendre en compte la biologie du virus, la spécificité tumorale et la capacité de porter des gènes thérapeutiques \cite{mcfadden2021}. De plus, la réponse immunitaire anticancéreuse varie considérablement entre les modèles animaux et humains, ce qui complique la prédiction de l'efficacité des traitements à base de virus oncolytiques chez l'homme.

Enfin, l'utilisation de virus oncolytiques en tant qu'adjuvants pour la vaccination anticancéreuse personnalisée représente une autre approche innovante. Cette stratégie, actuellement en phase de test clinique, utilise deux virus différents exprimant le même antigène tumoral pour amorcer et renforcer l'immunité antitumorale \cite{pubmed2021}. L'efficacité de cette vaccination dépend en partie de la sensibilité variable de chaque tumeur à l'infection par le virus oncolytique.

\section*{Conclusion}

La virothérapie en oncologie est un domaine en pleine expansion, avec des progrès significatifs et des défis persistants. La compréhension approfondie des interactions entre les virus, les tumeurs et le système immunitaire est essentielle pour optimiser cette forme de traitement et la rendre plus efficace.

\bibliographystyle{plain}
\bibliography{references}

\end{document}